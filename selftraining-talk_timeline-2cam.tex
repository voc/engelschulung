\documentclass[aspectratio=169]{beamer}

\usetheme{metropolis}

% ENCODING AND LANGUAGE
\usepackage[english]{babel}

\usepackage[utf8]{inputenc}     % Universal encoding
\usepackage[T1]{fontenc}        % Font encoding

% FONT
\usepackage{courier}            % Courier as \ttdefault
% \usepackage{psfrag}             % replace PostScript fonts

% OTHER HELPERS AND SETTINGS
% \usepackage{graphicx}           % Include graphics to document
% \usepackage{amsmath,amssymb,amstext}  % support for mathematics ttt

% \usepackage{listings}           % code listings
% \lstset{basicstyle=\footnotesize\ttfamily,breaklines=true}

% \usepackage{units}
\usepackage{siunitx}            % SI-Unit support

% TITLE PAGE SETTINGS
\title{A/V Angel Self-Training: Talk Timeline with 2 Cameras}
\subtitle{How a typical talk could be shot and cut}
% \date{\today \currenttime}
% \author{c3voc}
\institute{C3VOC
	\begin{flushright}
		\includegraphics[height=0.4\textheight]{images/qr-code.png}\\
		https://github.com/voc/engelschulung
	\end{flushright}
}

%% START OF DOCUMENT
\begin{document}
\maketitle

\begin{frame}{Usual Talk Timeline}
  \begin{itemize}
    \item Please be on time when your shift starts
    \item Get to know your fellow angels, check the camera and mixer
    \item Talk starts with an introduction by the herald
    \item Speaker starts talk
    \item Q\&A session
    \item Talk ends with "thank you" and applause
    \item Hand over to the next angels
  \end{itemize}
\end{frame}

% !TEX root = ../main.tex

\begin{frame}{Timeline - Preparations}
	\begin{block}{Cameras}
		\begin{itemize}
			\item Test your camera and tripod
			\item Check who is herald and speaker
			\item Camera 1: Get a closeup of the speaker
			\item Camera 2: Show the stage, ideally with herald and speaker
		\end{itemize}
	\end{block}
	\begin{block}{Mixer}
		\begin{itemize}
			\item Have camera 2 on program and be prepared to go live
			\item Signal that you're ready when the talk should start
		\end{itemize}
	\end{block}
\end{frame}

\begin{frame}{Timeline - Introduction}
	\begin{block}{Cameras}
		\begin{itemize}
			\item Cam 1: Follow the speaker, if necessay
			\item Cam 2: Keep the herald in frame
		\end{itemize}
	\end{block}
	\begin{block}{Mixer}
		\begin{itemize}
			\item Go live with Camera 2 as soon as the Herald starts
			\item Title slide can be shown during the introduction 
			\item Be prepared to switch to Lecture Mode or Camera 1 as soon as the speaker starts talking
		\end{itemize}
	\end{block}
\end{frame}

\begin{frame}{Timeline - Content}
	\begin{block}{Cameras}
		\begin{itemize}
			\item With intercom: Follow commands from your mixer and ask for time off, if you need to re-adjust something
			\item Cam 1: Follow the speaker with a closeup shot
			\item Cam 2: Get a medium shot of the speaker (for gestures and walking around)
		\end{itemize}
	\end{block}
	\begin{block}{Mixer}
		\begin{itemize}
			\item Show new slides as soon as they are keyed by the Speaker 
			\item Show slides long enough (read slide 2 times)
			\item Use lecture mode, if possible
			\item Use the camera in fullscreen, if there's action on the stage
			\item Try to plan ahead and anticipate the next actions by the speaker
		\end{itemize}
	\end{block}
\end{frame}

\begin{frame}{Timeline - Questions and Answers}
	\begin{block}{Cameras}
		\begin{itemize}
			\item Cam 1: Track the Speaker
			\item Cam 2: Show the herald as well
			\item Keep tracking, don't give up even if your shift ends soon
		\end{itemize}
	\end{block}
	\begin{block}{Mixer}
		\begin{itemize}
			\item Show whoever is talking on stage to the stream
			\item The "Thanks"-Slide can be shown from time to time
			\item Don't end too early
		\end{itemize}
	\end{block}
\end{frame}


\end{document}
