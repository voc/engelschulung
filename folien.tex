\documentclass[aspectratio=169]{beamer}
%\documentclass{beamer}

%%%%%%%%%%%%%%%%%%%%%%%%%%%%%%%%%%%%%%%%%%%%%%%%%%%%%%%%%%%%%%%%%%%%%%%%%%%%%%%%
%%%
%%% packages
%%%

%%%
%%% encoding and language set
%%%

%%% ngerman: language set to new-german
%\usepackage{nenglish}

%%% babel: language set (can cause some conflicts with package ngerman)
%%%        use it only for multi-language documents or non-german ones
%\usepackage[ngerman]{babel}

%%% inputenc: coding of german special characters
\usepackage[utf8]{inputenc}

%%% fontenc, ae, aecompl: coding of characters in PDF documents
\usepackage[T1]{fontenc}
\usepackage{ae,aecompl}

%%%
%%% technical packages
%%%

%%% amsmath, amssymb, amstext: support for mathematics ttt
%\usepackage{amsmath,amssymb,amstext}

%%% psfrag: replace PostScript fonts
\usepackage{psfrag}

%%% listings: include programming code
\usepackage{listings}
\usepackage{courier}
\lstset{basicstyle=\footnotesize\ttfamily,breaklines=true}

%%% units: technical units
\usepackage{units}

%%% tiefgestellte zahlen
\usepackage{subscript}


\usepackage[official]{eurosym}

\usetheme{metropolis}           % Use metropolis theme

\usepackage{xspace}


%%%
%%% PDF
%%%

\usepackage{ifpdf}

%%% Should be LAST usepackage-call!
%%% For docu on that, see reference on package `hyperref''
\ifpdf%   (definitions for using pdflatex instead of latex)

%%% graphicx: support for graphics
%\usepackage[pdftex]{graphicx}

\pdfcompresslevel=9

%%% hyperref (hyperlinks in PDF): for more options or more detailed
%%%          explanations, see the documentation of the hyperref-package
\usepackage[%
%%% general options
pdftex=true,      %% sets up hyperref for use with the pdftex program
%plainpages=false, %% set it to false, if pdflatex complains: `destination with same identifier already exists''
%
%%% extension options
backref,      %% adds a backlink text to the end of each item in the bibliography
pagebackref=false, %% if true, creates backward references as a list of page numbers in the bibliography
colorlinks=true,   %% turn on colored links (true is better for on-screen reading, false is better for printout versions)
%
%%% PDF-specific display options
bookmarks=true,          %% if true, generate PDF bookmarks (requires two passes of pdflatex)
bookmarksopen=false,     %% if true, show all PDF bookmarks expanded
bookmarksnumbered=false, %% if true, add the section numbers to the bookmarks
%pdfstartpage={1},        %% determines, on which page the PDF file is opened
pdfpagemode=None         %% None, UseOutlines (=show bookmarks), UseThumbs (show thumbnails), FullScreen
]{hyperref}


%%% provide all graphics (also) in this format, so you don't have
%%% to add the file extensions to the \includegraphics-command
%%% and/or you don't have to distinguish between generating
%%% dvi/ps (through latex) and pdf (through pdflatex)
\DeclareGraphicsExtensions{.pdf}

\else %else   (definitions for using latex instead of pdflatex)

\usepackage[dvips]{graphicx}

\DeclareGraphicsExtensions{.eps}

\usepackage[%
dvips,           %% sets up hyperref for use with the dvips driver
colorlinks=false %% better for printout version; almost every hyperref-extension is eliminated by using dvips
]{hyperref}

\fi


%%%%%%%%%%%%%%%%%%%%%%%%%%%%%%%%%%%%%%%%%%%%%%%%%%%%%%%%%%%%%%%%%%%%%%%
%%% Title stuff

\title{Angel Introduction: Camera and Video Mixer}
\date{\today \currenttime}
\author{sophie}
\institute{C3VOC}

%%%%%%%%%%%%%%%%%%%%%%%%%%%%%%%%%%%%%%%%%%%%%%%%%%%%%%%%%%%%%%%%%%%%%%%%%%%%%%%%
%%%
%%% begin document
%%%

\begin{document}

%\pagenumbering{roman} %% small roman page numbers

%%% include the title
% \thispagestyle{empty}  %% no header/footer (only) on this page
\maketitle

%%% start a new page and display the table of contents
\begin{frame}{Inhalt}
\tableofcontents
\end{frame}
%%% start a new page and display the list of figures
% \newpage
% \listoffigures

%%% start a new page and display the list of tables
% \newpage
% \listoftables

%%% display the main document on a new page 
\newpage

% \pagenumbering{arabic} %% normal page numbers (include it, if roman was used above)

%%%%%%%%%%%%%%%%%%%%%%%%%%%%%%%%%%%%%%%%%%%%%%%%%%%%%%%%%%%%%%%%%%%%%%%%%%%%%%%%
%%%
%%% begin main document



%notes
%%%%%%%%%%%%%%%%%%%%%%%%%%%%%


%TODO

%witchtige slides rausziehen

%verlinkung erstellen
%

%wo
%was
%allgemeine begriffe

%wichtige folien an mischer/kameraplätze legen



% Begriffsdefinition
% Western Style nicht 

% Konvention weit hinten

%%%%%%%%%%%%%%%%%%%%%%%%
% erweiterungen
%-----------------------




% Verbesserungen
% Voctogui in groß
% streambeobachter als mischerengel 
% Bildregie als Kommunikationstraining



%%%%%%%%%%%%%%%%%%%%%%%%%%%%%%%%%%%%%%%%%%%%%%%%%%%%%%%%%%%%%%%%%%%%%%
%%% Slides

\section{General Info}
\begin{frame}{General Info I}
	\begin{itemize}
		\item All talks get recorded and archived forever
		\item Consistent quality
		\item No postproduction of individual signals.
		\item Livestream content is the same as the one recorded and published
		\item Less mistakes $\Rightarrow$ better recodings.
		\item Stream observer shifts
		\item Difficult talks together with additional video director
	\end{itemize}
\end{frame}


\begin{frame}{General Info II}
	\begin{itemize}
		\item Introduction Meeting here
		\item Complete overview for all new angels
		\item Short diff for experienced ones
		\item Shift distribution every day 15:00 to 16:00 in Hall 6.
		\item Feedback loop and review at those meetings
		\item Slides available online: \texttt{https://streaming.selfnet.de/engelschulung.pdf}
	\end{itemize}
	\begin{figure} 
		\centering
		\includegraphics[height=0.4\textheight]{images/qrcode.png}
		\label{fig:qr1}
	\end{figure}
\end{frame}

\section{Angeltypes}
\begin{frame}{Angeltypes}
	\begin{itemize}
		\item Camera Angels
		\item Video Mixer Angels
		\item Stream Observing Angels
		\item A/V Technician
	\end{itemize}
\end{frame}

\begin{frame}{Camera Angels}
	\begin{itemize}
		\item Operate the fixed cameras in the lecture halls. 
		\item Usually, two video angels per lecture hall 
		\item Camera angels will communicate with the Video-Mixer-Angel via intercom,
		\item Get instrucions to shoot in certain ways. 
		\item Maintain good camera settings 
	\end{itemize}
\end{frame}

\begin{frame}{Video Mixer Angels}
	\begin{itemize}
		\item Switch the video feed between different sources. 
		\item Mixed video feed is used for both the live-stream and the recordings 
		\item You decide which picture, respectively source, is most interesting/important at each moment.
		\item Work proactively with camera angels through the intercom, 
		\item Challenging talks, with assistance from an external "image composition director" joining the intercom channel.
	\end{itemize}
\end{frame}

\begin{frame}{Stream Observing Angel}
	\begin{itemize}
		\item Open for all camera and mixing angels
		\item Reflecting the work of colleagues from an audience perspective.
		\item Examine streams for issues 
		\item Keep track of sequences appearing hard to consume or violating our rule set. 
		\item Positive and negative remarks 
		\item Constructive feedback 
		\item Instantly report severe issues like "there is no signal" to the VOC Helpdesk.
		\item Self evaluation and not meant as external monitoring. 
	\end{itemize}
\end{frame}

\begin{frame}{A/V Technician}
	\begin{itemize}
		\item 2nd level support in the lecture rooms. 
		\item Familiar with the equipment that is used 
		\item Able to fix (nearly) all the issues. 
		\item Longer shifts.
	\end{itemize}
\end{frame}

\section{Camera Hardware}
\begin{frame}{Hardware Camera Controls}
	\begin{columns}[T,onlytextwidth]
	\column{0.5\textwidth}
	\begin{figure} 
		\centering
		\def\svgwidth{0.9\textwidth}
		\import{images}{camera-controls.pdf.tex}
		\caption{Camera Controls}
		\label{fig:cc1}
	\end{figure}
	\column{0.5\textwidth}
	Cameras are in manual mode because of difficult lighting situation.
	\begin{description}
		\item[Left Ring] Focus - control sharpness of the image.
		\item[Middle Ring] Zoom - vary the focal length.
		\item[Right Ring] Iris - don't touch.
     \end{description}
\end{columns}
\end{frame}

\begin{frame}{Tripod Handle Controls}
	\begin{columns}[T,onlytextwidth]
	\column{0.5\textwidth}
	\begin{figure} 
		\centering
		\includegraphics[width=0.7\textwidth]{images/tripod-handle.jpeg}
		\caption{Tripod Handle}
		\label{fig:th1}
	\end{figure}

	\column{0.5\textwidth}
	Beware: various models in use.
	\begin{description}
		\item[Zoom Control] lever above red ring
		\item[Red Button] Start/stop recording, don't touch
		\item[Other Buttons] markings on the handle
    \end{description}
	\metroset{block=fill}
	\begin{alertblock}{Alert}
		    Saal 1 and Saal 2 have old tele zoom lenses. Left handle Focus. Right handle Zoom.
	\end{alertblock}

	\end{columns}
\end{frame}

\begin{frame}{Tripod}
	\begin{columns}[T,onlytextwidth]
	\column{0.4\textwidth}
	\begin{figure} 
		\centering
		\includegraphics[width=0.9\textwidth]{images/tripod-complete.png}
		\caption{Tripod}
		\label{fig:tp1}
	\end{figure}
	
	\column{0.6\textwidth}
	\begin{itemize}
			\item Should be level - check the water bubble.
			\item Variable brakes - can be adjusted to your needs.
			\item Tilt axis should be balanced, so that the camera doesn't tilt up or down on its own.
			\item Pan axis is needed all of the time. Set it so you can do smooth pans all over the stage.
		\end{itemize}
		\metroset{block=fill}
		\begin{alertblock}{Alert}
			Alert the A/V-Technician if something's wrong or misplaced.
		\end{alertblock}
	\end{columns}
\end{frame}

\begin{frame}{SD-Card Recording}
		\begin{itemize}
			\item Two SD-Cards in one camera each room
			\item Backup Recording
			\item Turn on Recording before first shift in the morning -> Red Dot somewhere in the Display.
			\item Control Recording Time remaining. 
		\end{itemize}
		\metroset{block=fill}
		\begin{alertblock}{Alert}
			Alert the A/V-Technichian if something's wrong or not running.
		\end{alertblock}
\end{frame}

\section{Camera Positions and Angles}
\begin{frame}{Map Saal 1 + Saal 2}
	\begin{figure} 
		\centering
		\includegraphics[height=0.9\textheight]{images/1-2-base-cameras.png}
		\label{fig:map1}
	\end{figure}
\end{frame}

\begin{frame}{Map Saal G}
	\begin{figure} 
		\centering
		\includegraphics[height=0.9\textheight]{images/g-base-cameras.png}
		\label{fig:mapg1}
	\end{figure}
\end{frame}

\begin{frame}{Map Saal 6}
	\begin{figure} 
		\centering
		\includegraphics[height=0.9\textheight]{images/6-base-cameras.png}
		\label{fig:map61}
	\end{figure}
\end{frame}

\section{Camera 1 - Closeup Camera}
\begin{frame}{Camera 1 - Closeup Camera}
		\begin{block}{Content}
			\begin{itemize}
				\item The Speaker is your best friend \\
				\item Keep them always in frame.
			\end{itemize}
		\end{block}
		
		\begin{block}{Framing}
			\begin{itemize}
				\item The upper part of their body + head + a bit of headroom. \\
				\item Stay close to his/her eyeline on the upper third line.
			\end{itemize}
		\end{block}

		\begin{alertblock}{Alerts}
			\begin{itemize}
				\item Anticipate movement. \\
				\item Leave some room where they want to move next. \\
				\item Needs lots of attention.
			\end{itemize}
		\end{alertblock}
\end{frame}

\begin{frame}{Camera 1 - Closeup Camera}
	Example Shots I
	\begin{figure} 
		\centering
		\includegraphics[width=0.7\textwidth]{images/closeup1.jpg}
		\caption{Good Closeup Shot}
		\label{fig:close1}
	\end{figure}
\end{frame}

\begin{frame}{Camera 1 - Closeup Camera}
	Example Shots II
	\begin{figure} 
		\centering
		\includegraphics[width=0.7\textwidth]{images/closeup2.jpg}
		\caption{Good Closeup in Supersource}
		\label{fig:close2}
	\end{figure}
\end{frame}

\begin{frame}{Camera 1 - Closeup Camera}
	Bad Shots I
	\begin{figure} 
		\centering
		\includegraphics[width=0.7\textwidth]{images/closeup-bad1.png}
		\caption{Half a head - not good.}
		\label{fig:closeb1}
	\end{figure}
\end{frame}

\begin{frame}{Camera 1 - Closeup Camera}
	Bad Shots II
	\begin{figure} 
		\centering
		\includegraphics[width=0.7\textwidth]{images/closeup-bad2.png}
		\caption{Too Far out for a good supersource image.}
		\label{fig:closeb2}
	\end{figure}
\end{frame}


\section{Camera 2 - Medium Camera}
\begin{frame}{Camera 2 - Medium Camera}
		\begin{block}{Content}
			\begin{itemize}
				\item Context around the speaker \\
				\item If there are two or more speakers choose the other one - \textbf{COMMUNICATE}
			\end{itemize}
		\end{block}
		
		\begin{block}{Framing}
			\begin{itemize}
				\item Speaker from Head to Toes \\
				\item Stay close to his/her eyeline on the upper third line.
			\end{itemize}
		\end{block}

		\begin{alertblock}{Alerts}
			\begin{itemize}
				\item Anticipate movement. \\
				\item Leave some room where they want to move next. \\
				\item Fallback Camera if the Closeup Camera can't keep up.
			\end{itemize}
		\end{alertblock}
\end{frame}

\begin{frame}{Camera 2 - Medium Camera}
	\textbf{Good Shots I}
	\begin{figure} 
		\centering
		\includegraphics[width=0.7\textwidth]{images/medium1.png}
		\caption{Good Context image.}
		\label{fig:medium1}
	\end{figure}
\end{frame}

\begin{frame}{Camera 2 - Medium Camera}
	\textbf{Good Shots II}
	\begin{figure} 
		\centering
		\includegraphics[width=0.7\textwidth]{images/medium2.png}
		\caption{Two Speakers.}
		\label{fig:medium2}
	\end{figure}
\end{frame}

\begin{frame}{Camera 2 - Medium Camera}
	\textbf{Bad Shots I}
	\begin{figure} 
		\centering
		\includegraphics[width=0.7\textwidth]{images/medium-bad1.png}
		\caption{Too much audience.}
		\label{fig:mediumb1}
	\end{figure}
\end{frame}

\begin{frame}{Camera 2 - Medium Camera}
	\textbf{Bad Shots II}
	\begin{figure} 
		\centering
		\includegraphics[width=0.7\textwidth]{images/medium-bad2.png}
		\caption{Too much audience.}
		\label{fig:mediumb1}
	\end{figure}
\end{frame}

\section{Camera 3 - Wide Shot}

\begin{frame}{Camera 3 - Wide Shot}
		\begin{block}{Content}
			\begin{itemize}
				\item Complete Lecture Hall in Saal 1 and 2 \\
			\end{itemize}
		\end{block}
		
		\begin{block}{Framing}
			\begin{itemize}
				\item Covers the whole stage.
				\item A bit of small audience for context.
				\item Statically set.
			\end{itemize}
		\end{block}

		\begin{alertblock}{Alerts}
			\begin{itemize}
				\item Needs no attention. 
				\item Fallback Camera if all else fails or standing ovations erupt.
			\end{itemize}
		\end{alertblock}
\end{frame}

\section{Video Mixer Tools}

\begin{frame}{Dual Setup}
	Two different Setups are in use at this congress. 
	Both setups offer similar possibilities. 
	You can work with both of them.

	\begin{columns}[T,onlytextwidth]
	\column{0.4\textwidth}
		\metroset{block=fill}
		\begin{exampleblock}{Hardware Video Mixer}
			\begin{itemize}
				\item Saal 1
				\item Saal 2
				\item Specialized Hardware Panel
			\end{itemize}
		\end{exampleblock}

	\column{0.4\textwidth}
		\metroset{block=fill}
		\begin{exampleblock}{Software Video Mixer}
			\begin{itemize}
				\item Saal G
				\item Saal 6
				\item Sendezentrum
				\item Laptop as control surface.
			\end{itemize}
		\end{exampleblock}
	\end{columns}

\end{frame}

\begin{frame}{Hardware Video - Mixer Controls}
	\begin{columns}[T,onlytextwidth]
	\column{0.5\textwidth}
	\begin{figure} 
		\centering
		\def\svgwidth{0.9\textwidth}
		\import{images}{atem-control-panel.pdf.tex"}
		\caption{ATEM 1M/E Panel}
		\label{fig:1me1}
	\end{figure}
	\column{0.5\textwidth}
	\begin{description}
		\item[Preview PRV Select] Green row, lower left 
		\item[Program PGM Select] Red row above green row.
		\item[Cut Button] Swap selections of PRV and PGM Rows.
		\item[Crossfade Bar] Smooth transition between PRV and PGM.
     \end{description}
	\end{columns}
\end{frame}

\begin{frame}{Hardware Video Mixer - Screen Layout}
	\begin{columns}[T,onlytextwidth]
	\column{0.5\textwidth}
	\begin{figure} 
		\centering
		\includegraphics[width=0.7\textwidth]{images/atem-multiview.png}
		\caption{ATEM Multi View}
		\label{fig:atem2}
	\end{figure}

	\column{0.5\textwidth}
	\begin{description}
		\item[Top Left] Preview, your preview to check feeds before going live 
		\item[Top Right] Program, what everyone on the internet sees.
		\item[Bottom two Rows] Up to eight video Feeds.
     \end{description}
	\end{columns}
\end{frame}


\begin{frame}{Software Video Mixer - Controls}
	\begin{columns}[T,onlytextwidth]
	\column{0.5\textwidth}
	\begin{figure} 
		\centering
		\includegraphics[width=1\textwidth]{images/voctomix.png}
		\caption{Voctogui}
		\label{fig:voctogui1}
	\end{figure}

	\column{0.5\textwidth}
	\begin{description}
		\item[Previews] Small images on the left 
		\item[Program] Large, middle, what everyone on the internet sees.
		\item[Composition] Top row.
		\item[Blue] Select A
		\item[Red] Select B
		\item[Stream Blank] For breaks when nothing should be streamed.
     \end{description}
	\end{columns}
\end{frame}

\begin{frame}{Software Video Mixer - Voctogui}
	\begin{figure} 
		\centering
		\includegraphics[width=.9\textwidth]{images/voctomix.png}
		\caption{Voctogui}
		\label{fig:voctogui2}
	\end{figure}
\end{frame}

\section{Video Mixing Guidelines}
\begin{frame}{Mixing Guidelines - Hard Rules}
	\begin{itemize}
		\item \textbf{All} you are doing is \textbf{recorded} and will be published. \alert{\textbf{Don't make mistakes.}}
		\item The Audience is \textbf{not to be filmed}. Cut away if faces of people not on the stage appear.
		\item \textbf{Slides are important}
		\item Slides stay on till the text has been read \textbf{twice}.
		\item Show new slides \textbf{immediately}.
	\end{itemize}
	\begin{exampleblock}{Hint}
		Fast-paced presentations with lots of slides are easier to handle with the supersource.
	\end{exampleblock}
\end{frame}


\begin{frame}{Mixing Guidelines - Softer Hints}
	\begin{itemize}
		\item Start early – opening announcements of the Herald are a good start. Their introduction has to be in the recording and on stream.
		\item Open wide – Structure the beginning of a talk with shots that set the stage
		\item The slides in fullscreen – you’re dealing with a very small screen. Text has to be readable
		\item Show gestures – medium-close-up that follows the speakers eye-line
		\item Don’t be too cutty – Pace your videos temperately. Do not cut too often.
		\item Don't end too early – All questions and answers have to be recorded. The herald ends the talk, not the mixer angel.
	\end{itemize}
	\begin{exampleblock}{Hints}
		Leave lots of room at the start and end of a talk. 
		Cut away from the infobeamer before the Herald starts with announcements. 
		Cut to the infobeamer only after the last applause has finished.
	\end{exampleblock}
\end{frame}

\begin{frame}{Mixing Guidelines - Communication}
	\begin{itemize}
		\item Communication is key
		\item Partyline intercom in every room
		\item Mixer Angel requests pictures from Camera Angels and announcs their next steps
		\item Camera angels offer good pictures
		\item Work together, say what you want to do and what doesn't work.
	\end{itemize}
\end{frame}

\section{Timeline of a Talk}

\begin{frame}{Timeline of a typical talk}
	\begin{enumerate}
		\item Preparations beforehand
		\item Announcements and Introduction
		\item Content
		\item Questions and Answers
		\item Ending
	\end{enumerate}
\end{frame}

\begin{frame}{Timeline - Preparations beforehand}
	\begin{block}{Cameras}
		\begin{itemize}
			\item Get to know your fellow angels.
			\item Test the intercom.
			\item Test your camera and settings.
			\item Look on Stage who will be Herald ans Speaker.
			\item Camera 1: Get a closeup of the speaker.
			\item Camera 2: Get a head to toes shot of the herald.
			\item Both cameras start tracking their persons.
		\end{itemize}
	\end{block}
	\begin{block}{Mixer}
		\begin{itemize}
			\item Check Slides and adjust supersource to 16:9 or 4:3.
			\item Have Camera 2 on Preview.
			\item Talk to your cameras via the intercom.
		\end{itemize}
	\end{block}
\end{frame}

\begin{frame}{Timeline - Announcements and Introduction}
	\begin{block}{Cameras}
		\begin{itemize}
			\item Camera 1: Get a closeup of the speaker.
			\item Camera 2: Get a head to toes shot of the herald.
			\item Both cameras track their persons.
		\end{itemize}
	\end{block}
	
	\begin{block}{Mixer}
		\begin{enumerate}
			\item Go live with Camera 2 as soon as the Herald starts.
			\item Title slide can be shown during the introduction 
			\item Put Camera 1 on Preview.
			\item Camera 1 live as soon as the Speaker starts talking
		\end{enumerate}
	\end{block}
\end{frame}

\begin{frame}{Timeline - Content}
	\begin{block}{Cameras}
		\begin{itemize}
			\item Follow commands from the Mixer
			\item Ask for time off if you have to readjust zoom, focus or anything else that should noc be in the recording.
		\end{itemize}
	\end{block}
	
	\begin{block}{Mixer}
		\begin{itemize}
			\item Show new slides as soon as they are keyed by the Speaker 
			\item Show Camera 2 when the Speaker starts walking and gesturing
			\item Call out your actions and intentions via intercom
			\item Plan ahead. Which picture should be shown in 30 seconds?
			\item Keep your available pictures diverse. Both cameras on medium closeup don't make sense.
		\end{itemize}
	\end{block}
\end{frame}

\begin{frame}{Timeline - Questions and Answers}
	\begin{block}{Cameras}
		\begin{itemize}
			\item Camera 1: Track the Speaker
			\item Camera 2: Track the Herald or both if they are close
			\item Keep tracking, don't give up even if your shift ends soon.
		\end{itemize}
	\end{block}
	
	\begin{block}{Mixer}
		\begin{itemize}
			\item Show whoever is talking on stage to the stream.
			\item The "Thanks"-Slide can be shown from time to time.
			\item Don't end too early.
		\end{itemize}
	\end{block}
\end{frame}


\begin{frame}{Questions?}
Contact me on congress via irc, voc-lounge on hackint.
\end{frame}

\end{document}


%%% End
