\documentclass[aspectratio=169]{beamer}
\usepackage{polyglossia}    
\setmainlanguage{english}
\usepackage[utf8]{inputenc}
\usepackage[T1]{fontenc}
\usepackage{ae,aecompl}
\usepackage{psfrag}
\usepackage{listings}
\usepackage{courier}
\lstset{basicstyle=\footnotesize\ttfamily,breaklines=true}
\usepackage{units}
\usepackage[official]{eurosym}
\usetheme{metropolis}           % Use metropolis theme
\usepackage{xspace}
\usepackage{qrcode}


\title{Angel Introduction: A/V Bunny}
\author{sophie \& jwacalex}
\institute{C3VOC}


\begin{document}

\maketitle


\begin{frame}{Inhalt}
\tableofcontents
\end{frame}


\section{General Info}
\begin{frame}{General Info I}
	\begin{itemize}
		\item All talks get recorded and archived forever
		\item Consistent quality
		\item No postproduction of individual signals.
		\item Livestream content is the same as the one recorded and published
		\item Less mistakes $\Rightarrow$ better recodings.
		\item Stream observer shifts
		\item Difficult talks together with additional video director
	\end{itemize}
\end{frame}


\begin{frame}{General Info II}
	\begin{itemize}
		\item Introduction Meeting here
		\item Complete overview for all new angels
		\item Short diff for experienced ones
		\item Shift distribution every day 17:00 in CCL 11.
		\item Feedback loop and review at those meetings
		\item Slides available online: \texttt{https://github.com/jwacalex/engelschulung/blob/eh19/folien.pdf}
	\end{itemize}
	\begin{figure} 
		\centering
		\qrset{link, height=5cm}
		\qrcode{https://github.com/jwacalex/engelschulung/blob/eh19/folien.pdf}
	\end{figure}
\end{frame}

\section{Angeltypes}
\begin{frame}{Angeltypes}
	\begin{itemize}
		\item Camera Angels
		\item Video Mixer Angels
		\item Stream Observing Angels
		\item A/V Technician
		\item Stage Manager
	\end{itemize}
\end{frame}

\begin{frame}{Camera Angels}
	\begin{itemize}
		\item Operate the fixed cameras in the lecture halls. 
		\item Usually, two video angels per lecture hall 
		\item Camera angels will communicate with the Video-Mixer-Angel via intercom,
		\item Get instrucions to shoot in certain ways. 
		\item Maintain good camera settings 
	\end{itemize}
\end{frame}

\begin{frame}{Video Mixer Angels}
	\begin{itemize}
		\item Switch the video feed between different sources. 
		\item Mixed video feed is used for both the live-stream and the recordings 
		\item You decide which picture, respectively source, is most interesting/important at each moment.
		\item Work proactively with camera angels through the intercom, 
		\item Challenging talks, with assistance from an external "image composition director" joining the intercom channel.
	\end{itemize}
\end{frame}

\begin{frame}{Stream Observing Angel}
	\begin{itemize}
		\item Open for all camera and mixing angels
		\item Reflecting the work of colleagues from an audience perspective.
		\item Examine streams for issues 
		\item Keep track of sequences appearing hard to consume or violating our rule set. 
		\item Positive and negative remarks 
		\item Constructive feedback 
		\item Instantly report severe issues like "there is no signal" to the VOC Helpdesk.
		\item Self evaluation and not meant as external monitoring. 
	\end{itemize}
\end{frame}

\begin{frame}{A/V Technician}
	\begin{itemize}
		\item 2nd level support in the lecture rooms. 
		\item is responsible for A/V Angels
		\item Familiar with the equipment that is used 
		\item Able to fix (nearly) all the issues. 
		\item is on intercom
	\end{itemize}
\end{frame}


\begin{frame}{Stage Manager}
\begin{itemize}
	\item is responsible for the lecture hall, especially
	\begin{itemize}
		\item crowd control
		\item time keeping
		\item last minute issues 
	\end{itemize}
	\item carries the radio for emergency communication
\end{itemize}
\end{frame}

\begin{frame}{A/V Technician \& Stage Manager}
\begin{itemize}
	\item have the same shift slots (4h) together
	\item Stage Manager is communication gateway to heralds
\end{itemize}
\end{frame}


\section{Camera Hardware}
\begin{frame}{Hardware Camera Controls}
	\begin{columns}[T,onlytextwidth]
		\column{0.5\textwidth}
		\begin{figure} 
			\centering
			%\def\svgwidth{0.9\textwidth}
			%\import{images}{camera-controls.pdf.tex}
			\caption{Camera Controls}
		\end{figure}
		\column{0.5\textwidth}
		Cameras are in manual mode because of difficult lighting situation.
		\begin{description}
			\item[Left Ring] Focus - control sharpness of the image.
			\item[Middle Ring] Zoom - vary the focal length.
			\item[Right Ring] Iris - don't touch.
		\end{description}
	\end{columns}
\end{frame}

\begin{frame}{Tripod Handle Controls}
	\begin{columns}[T,onlytextwidth]
	\column{0.5\textwidth}
	\begin{figure} 
		\centering
		\includegraphics[width=0.7\textwidth]{images/tripod-handle.jpeg}
		\caption{Tripod Handle}
	\end{figure}

	\column{0.5\textwidth}
	Beware: various models in use.
	\begin{description}
		\item[Zoom Control] lever above red ring
		\item[Red Button] Start/stop recording, don't touch
		\item[Other Buttons] markings on the handle
    \end{description}
	\metroset{block=fill}
	\begin{alertblock}{Alert}
		    Saal 1 and Saal 2 have old tele zoom lenses. Left handle Focus. Right handle Zoom.
	\end{alertblock}

	\end{columns}
\end{frame}

\begin{frame}{Tripod}
	\begin{columns}[T,onlytextwidth]
	\column{0.4\textwidth}
	\begin{figure} 
		\centering
		\includegraphics[width=0.9\textwidth]{images/tripod-complete.png}
		\caption{Tripod}
	\end{figure}
	
	\column{0.6\textwidth}
	\begin{itemize}
			\item Should be level - check the water bubble.
			\item Variable brakes - can be adjusted to your needs.
			\item Tilt axis should be balanced, so that the camera doesn't tilt up or down on its own.
			\item Pan axis is needed all of the time. Set it so you can do smooth pans all over the stage.
		\end{itemize}
		\metroset{block=fill}
		\begin{alertblock}{Alert}
			Alert the A/V-Technician if something's wrong or misplaced.
		\end{alertblock}
	\end{columns}
\end{frame}

\begin{frame}{SD-Card Recording}
		\begin{itemize}
			\item Two SD-Cards in one camera each room
			\item Backup Recording
			\item Turn on Recording before first shift in the morning -> Red Dot somewhere in the Display.
			\item Control Recording Time remaining. 
		\end{itemize}
		\metroset{block=fill}
		\begin{alertblock}{Alert}
			Alert the A/V-Technichian if something's wrong or not running.
		\end{alertblock}
\end{frame}

\section{Camera Positions and Angles}
\begin{frame}{Map Saal 1 + Saal 2}
	\begin{figure} 
		\centering
		\includegraphics[height=0.9\textheight]{images/1-2-base-cameras.png}
	\end{figure}
\end{frame}

\begin{frame}{Map Saal G}
	\begin{figure} 
		\centering
		\includegraphics[height=0.9\textheight]{images/g-base-cameras.png}
	\end{figure}
\end{frame}

\begin{frame}{Map Saal 6}
	\begin{figure} 
		\centering
		\includegraphics[height=0.9\textheight]{images/6-base-cameras.png}
	\end{figure}
\end{frame}

\section{Camera 1 - Closeup Camera}
\begin{frame}{Camera 1 - Closeup Camera}
		\begin{block}{Content}
			\begin{itemize}
				\item The Speaker is your best friend \\
				\item Keep them always in frame.
			\end{itemize}
		\end{block}
		
		\begin{block}{Framing}
			\begin{itemize}
				\item The upper part of their body + head + a bit of headroom. \\
				\item Stay close to his/her eyeline on the upper third line.
			\end{itemize}
		\end{block}

		\begin{alertblock}{Alerts}
			\begin{itemize}
				\item Anticipate movement. \\
				\item Leave some room where they want to move next. \\
				\item Needs lots of attention.
			\end{itemize}
		\end{alertblock}
\end{frame}

\begin{frame}{Camera 1 - Closeup Camera}
	Example Shots I
	\begin{figure} 
		\centering
		\includegraphics[width=0.7\textwidth]{images/closeup1.jpg}
		\caption{Good Closeup Shot}
	\end{figure}
\end{frame}

\begin{frame}{Camera 1 - Closeup Camera}
	Example Shots II
	\begin{figure} 
		\centering
		\includegraphics[width=0.7\textwidth]{images/closeup2.jpg}
		\caption{Good Closeup in Supersource}
	\end{figure}
\end{frame}

\begin{frame}{Camera 1 - Closeup Camera}
	Bad Shots I
	\begin{figure} 
		\centering
		\includegraphics[width=0.7\textwidth]{images/closeup-bad1.png}
		\caption{Half a head - not good.}
	\end{figure}
\end{frame}

\begin{frame}{Camera 1 - Closeup Camera}
	Bad Shots II
	\begin{figure} 
		\centering
		\includegraphics[width=0.7\textwidth]{images/closeup-bad2.png}
		\caption{Too Far out for a good supersource image.}
	\end{figure}
\end{frame}


\section{Camera 2 - Medium Camera}
\begin{frame}{Camera 2 - Medium Camera}
		\begin{block}{Content}
			\begin{itemize}
				\item Context around the speaker \\
				\item If there are two or more speakers choose the other one - \textbf{COMMUNICATE}
			\end{itemize}
		\end{block}
		
		\begin{block}{Framing}
			\begin{itemize}
				\item Speaker from Head to Toes \\
				\item Stay close to his/her eyeline on the upper third line.
			\end{itemize}
		\end{block}

		\begin{alertblock}{Alerts}
			\begin{itemize}
				\item Anticipate movement. \\
				\item Leave some room where they want to move next. \\
				\item Fallback Camera if the Closeup Camera can't keep up.
			\end{itemize}
		\end{alertblock}
\end{frame}

\begin{frame}{Camera 2 - Medium Camera}
	\textbf{Good Shots I}
	\begin{figure} 
		\centering
		\includegraphics[width=0.7\textwidth]{images/medium1.png}
		\caption{Good Context image.}
	\end{figure}
\end{frame}

\begin{frame}{Camera 2 - Medium Camera}
	\textbf{Good Shots II}
	\begin{figure} 
		\centering
		\includegraphics[width=0.7\textwidth]{images/medium2.png}
		\caption{Two Speakers.}
	\end{figure}
\end{frame}

\begin{frame}{Camera 2 - Medium Camera}
	\textbf{Bad Shots I}
	\begin{figure} 
		\centering
		\includegraphics[width=0.7\textwidth]{images/medium-bad1.png}
		\caption{Too much audience.}
	\end{figure}
\end{frame}

\begin{frame}{Camera 2 - Medium Camera}
	\textbf{Bad Shots II}
	\begin{figure} 
		\centering
		\includegraphics[width=0.7\textwidth]{images/medium-bad2.png}
		\caption{Too much audience.}
	\end{figure}
\end{frame}

\section{Camera 3 - Wide Shot}

\begin{frame}{Camera 3 - Wide Shot}
		\begin{block}{Content}
			\begin{itemize}
				\item Complete Lecture Hall in Saal 1 and 2 \\
			\end{itemize}
		\end{block}
		
		\begin{block}{Framing}
			\begin{itemize}
				\item Covers the whole stage.
				\item A bit of small audience for context.
				\item Statically set.
			\end{itemize}
		\end{block}

		\begin{alertblock}{Alerts}
			\begin{itemize}
				\item Needs no attention. 
				\item Fallback Camera if all else fails or standing ovations erupt.
			\end{itemize}
		\end{alertblock}
\end{frame}

\section{Video Mixer Tools}

\begin{frame}{Dual Setup}
	Two different Setups are in use at this congress. 
	Both setups offer similar possibilities. 
	You can work with both of them.

	\begin{columns}[T,onlytextwidth]
	\column{0.4\textwidth}
		\metroset{block=fill}
		\begin{exampleblock}{Hardware Video Mixer}
			\begin{itemize}
				\item Saal 1
				\item Saal 2
				\item Specialized Hardware Panel
			\end{itemize}
		\end{exampleblock}

	\column{0.4\textwidth}
		\metroset{block=fill}
		\begin{exampleblock}{Software Video Mixer}
			\begin{itemize}
				\item Saal G
				\item Saal 6
				\item Sendezentrum
				\item Laptop as control surface.
			\end{itemize}
		\end{exampleblock}
	\end{columns}

\end{frame}

\begin{frame}{Software Video Mixer - Controls}
	\begin{columns}[T,onlytextwidth]
	\column{0.5\textwidth}
	\begin{figure} 
		\centering
		\includegraphics[width=1\textwidth]{images/voctomix.png}
		\caption{Voctogui}
		\label{fig:voctogui1}
	\end{figure}

	\column{0.5\textwidth}
	\begin{description}
		\item[Previews] Small images on the left 
		\item[Program] Large, middle, what everyone on the internet sees.
		\item[Composition] Top row.
		\item[Blue] Select A
		\item[Red] Select B
		\item[Stream Blank] For breaks when nothing should be streamed.
     \end{description}
	\end{columns}
\end{frame}

\begin{frame}{Software Video Mixer - Voctogui}
	\begin{figure} 
		\centering
		\includegraphics[width=.9\textwidth]{images/voctomix.png}
		\caption{Voctogui}
	\end{figure}
\end{frame}

\section{Video Mixing Guidelines}
\begin{frame}{Mixing Guidelines - Hard Rules}
	\begin{itemize}
		\item \textbf{All} you are doing is \textbf{recorded} and will be published. \alert{\textbf{Don't make mistakes.}}
		\item The Audience is \textbf{not to be filmed}. Cut away if faces of people not on the stage appear.
		\item \textbf{Slides are important}
		\item Slides stay on till the text has been read \textbf{twice}.
		\item Show new slides \textbf{immediately}.
	\end{itemize}
	\begin{exampleblock}{Hint}
		Fast-paced presentations with lots of slides are easier to handle with the supersource.
	\end{exampleblock}
\end{frame}


\begin{frame}{Mixing Guidelines - Softer Hints}
	\begin{itemize}
		\item Start early – opening announcements of the Herald are a good start. Their introduction has to be in the recording and on stream.
		\item Open wide – Structure the beginning of a talk with shots that set the stage
		\item The slides in fullscreen – you’re dealing with a very small screen. Text has to be readable
		\item Show gestures – medium-close-up that follows the speakers eye-line
		\item Don’t be too cutty – Pace your videos temperately. Do not cut too often.
		\item Don't end too early – All questions and answers have to be recorded. The herald ends the talk, not the mixer angel.
	\end{itemize}
	\begin{exampleblock}{Hints}
		Leave lots of room at the start and end of a talk. 
		Cut away from the infobeamer before the Herald starts with announcements. 
		Cut to the infobeamer only after the last applause has finished.
	\end{exampleblock}
\end{frame}

\begin{frame}{Mixing Guidelines - Communication}
	\begin{itemize}
		\item Communication is key
		\item Partyline intercom in every room
		\item Mixer Angel requests pictures from Camera Angels and announcs their next steps
		\item Camera angels offer good pictures
		\item Work together, say what you want to do and what doesn't work.
	\end{itemize}
\end{frame}

\section{Timeline of a Talk}

\begin{frame}{Timeline of a typical talk}
	\begin{enumerate}
		\item Preparations beforehand
		\item Announcements and Introduction
		\item Content
		\item Questions and Answers
		\item Ending
	\end{enumerate}
\end{frame}

\begin{frame}{Timeline - Preparations beforehand}
	\begin{block}{Cameras}
		\begin{itemize}
			\item Get to know your fellow angels.
			\item Test the intercom.
			\item Test your camera and settings.
			\item Look on Stage who will be Herald ans Speaker.
			\item Camera 1: Get a closeup of the speaker.
			\item Camera 2: Get a head to toes shot of the herald.
			\item Both cameras start tracking their persons.
		\end{itemize}
	\end{block}
	\begin{block}{Mixer}
		\begin{itemize}
			\item Check Slides and adjust supersource to 16:9 or 4:3.
			\item Have Camera 2 on Preview.
			\item Talk to your cameras via the intercom.
		\end{itemize}
	\end{block}
\end{frame}

\begin{frame}{Timeline - Announcements and Introduction}
	\begin{block}{Cameras}
		\begin{itemize}
			\item Camera 1: Get a closeup of the speaker.
			\item Camera 2: Get a head to toes shot of the herald.
			\item Both cameras track their persons.
		\end{itemize}
	\end{block}
	
	\begin{block}{Mixer}
		\begin{enumerate}
			\item Go live with Camera 2 as soon as the Herald starts.
			\item Title slide can be shown during the introduction 
			\item Put Camera 1 on Preview.
			\item Camera 1 live as soon as the Speaker starts talking
		\end{enumerate}
	\end{block}
\end{frame}

\begin{frame}{Timeline - Content}
	\begin{block}{Cameras}
		\begin{itemize}
			\item Follow commands from the Mixer
			\item Ask for time off if you have to readjust zoom, focus or anything else that should noc be in the recording.
		\end{itemize}
	\end{block}
	
	\begin{block}{Mixer}
		\begin{itemize}
			\item Show new slides as soon as they are keyed by the Speaker 
			\item Show Camera 2 when the Speaker starts walking and gesturing
			\item Call out your actions and intentions via intercom
			\item Plan ahead. Which picture should be shown in 30 seconds?
			\item Keep your available pictures diverse. Both cameras on medium closeup don't make sense.
		\end{itemize}
	\end{block}
\end{frame}

\begin{frame}{Timeline - Questions and Answers}
	\begin{block}{Cameras}
		\begin{itemize}
			\item Camera 1: Track the Speaker
			\item Camera 2: Track the Herald or both if they are close
			\item Keep tracking, don't give up even if your shift ends soon.
		\end{itemize}
	\end{block}
	
	\begin{block}{Mixer}
		\begin{itemize}
			\item Show whoever is talking on stage to the stream.
			\item The "Thanks"-Slide can be shown from time to time.
			\item Don't end too early.
		\end{itemize}
	\end{block}
\end{frame}

\section{Orga} 		
\begin{frame}{Shift Distribution}		% Alex
\begin{itemize}
	\item Video Mixer shifts will be distributed in the angel system
	\item Sign up to the shifts you want to take
	\item Talks with special requirements might be handled by VOC
\end{itemize} 
\end{frame}
% mündlich: 	\item Everyone should do at least one shift per day

\subsection{Contacts}			% Alex
\begin{frame}{Who to Contact?}
\begin{itemize}
	\item General Issues - VOC \textbf{1600}
	\item Organizational or social problems / Angels - jwacalex - DECT \textbf{5523}
	\item General Angel Topics - Heaven - DECT \textbf{1023}
	\item Unable to find right person for issue - VOC Helpdesk \textbf{1600}
	\item We might need to call you. Please have your DECT (or UMTS) number in the Engelsystem! If you don't have a number yet, go to 
	\textcolor{blue}{\textbf{eventphone.de}} and get one. 
\end{itemize} 
\end{frame}

\begin{frame}{Questions?}
Contact us on EH19 via irc, voc-lounge on hackint.
\end{frame}

\end{document}


%%% End
